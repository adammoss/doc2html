\chapter{covariant derivatives}

The best way to think about gauge invariance is through the notion of a covariant derivative. The reason that the original Dirac equation does not have a local U(1) symmetry is that when 
we take $\psi\rightarrow e^{-i\Lambda(x)}\psi$, $\psi$ and $\del_\mu\psi$ transform differently. What we really want is some derivative, $D_\mu$, that transforms as
\ba
D_\mu\psi\rightarrow e^{-i\Lambda(x)}D_\mu\psi,
\ea
because then the derivative piece of the enhanced Dirac equation (i.e. the $\gamma^\mu D_\mu\psi$ piece) will transform in the same way as the mass term. This is just what we have 
constructed,
\ba
\label{eq:covDeriv}
D_\mu=\del_\mu-iq A_\mu
\ea
and we have
\ba
\psi'&=&e^{-i\Lambda(x)}\psi,\\
A'_\mu&=&A_\mu+\del_\mu\alpha\\
(D_\mu\psi)'&=&e^{-i\Lambda(x)}(D_\mu\psi).
\ea
We have therefore constructed a derivative that transforms in the same way as the thing it is differentiating. This is important because since we now that $(D_\mu\psi)$ transforms with a 
phase $e^{-i\Lambda(x)}$, then the covariant derivative of this will also transform with a phase $e^{-i\Lambda(x)}$. This extends to multiple covariant derivatives, i.e. all of $\psi$, 
$D_\mu\psi$, $D_\nu D_\mu\psi$, $D_\rho D_\nu D_\mu\psi$,...  transform with the same phase factor $e^{-i\Lambda(x)}$.

With this covariant derivative we are now in a position to construct new objects, the most important of which is the {\it field strength}, $F_{\mu\nu}$. This is defined by
\ba
\label{eq:U1FieldStrength}
\left[D_\mu,D_\nu\right]\psi=-iqF_{\mu\nu}\psi.
\ea
This is a nice quantity because it is gauge invariant, and so could be of physical relevance. To see that it is gauge invariant we note that as $D_\mu\psi$ transforms covariantly, so 
does $D_\nu D_\mu\psi$, that is one of the strengths of using covariant derivatives. This implies that $\left[D_\mu,D_\nu\right]\psi$ also transforms covariantly (i.e. 
$\left[D_\mu,D_\nu\right]\psi\rightarrow e^{-i\Lambda(x)}\left[D_\mu,D_\nu\right]\psi$). But, the right-hand-side of (\ref{eq:U1FieldStrength}) also picks up a $e^{-i\Lambda(x)}$, 
meaning that $F_{\mu\nu}$ must be unchanged, $F'_{\mu\nu}=F_{\mu\nu}$. 

To be more explicit note that
\ba
\psi'&=&U\psi,\quad(D_\mu\psi)'=U(D_\mu\psi),\quad(D_\mu D_\nu\psi)'=U(D_\mu D_\nu\psi)
\ea
where $U=e^{-i\Lambda(x)}$. So, the primed field strength is just
\ba
(\left[D_\mu,D_\nu\right]\psi)'&=&-iqF'_{\mu\nu}\psi'\\
U\left[D_\mu,D_\nu\right]\psi&=&-iqF'_{\mu\nu}U\psi\\
U(-iqF_{\mu\nu}\psi)&=&-iqF'_{\mu\nu}U\psi\\
UF_{\mu\nu}\psi&=&F'_{\mu\nu}U\psi
\ea
Now, this is required to hold for all $\psi$ so
\ba
UF_{\mu\nu}&=&F'_{\mu\nu}U
\ea
and now note that $UU^\dagger=1=U^\dagger$ so multiply on the right by $U^\dagger$.
\ba
UF_{\mu\nu}U^\dagger&=&F'_{\mu\nu}
\ea
then note that in this case the gauge theory is Abelian, i.e. U(1), so the $U$, $U^\dagger$ are just numbers and can move through the $F_{\mu\nu}$ to cancel each other
\ba
F_{\mu\nu}&=&F'_{\mu\nu}
\ea
This proves the gauge invariance of the field strength. Now we need to see what the field strength actually is, to do this, just calculate it
\ba
-iqF_{\mu\nu}\psi&=&D_\mu D_\nu\psi-D_\nu D_\mu\psi\\
  &=&(\del_\mu-iqA_\mu)(\del_\nu\psi-iqA_\nu\psi)-(\mu\leftrightarrow\nu)\\
  &\vdots&\\
  &=&-iq(\del_\mu A_\nu-\del_\nu A_\mu)\psi
\ea
so we find
\ba
F_{\mu\nu}&=&\del_\mu A_\nu-\del_\nu A_\mu
\ea
and now we are nearly there, just a few examples should convince you that $F_{\mu\nu}$ is nothing more than a re-packaging of the electric and magnetic fields.

e.g. $F_{01}=\del_0 A_1-\del_1 A_0=\dot A_x+\del_x\phi=-E^x$.

e.g. $F_{12}=\del_1 A_2-\del_2 A_1=\del_x A_y-\del_y A_x=B^z$.

It is a useful excercise to check the other components.


