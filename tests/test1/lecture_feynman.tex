\chapter{diagramatic expansion}

As is common with such perturbative expansions, it is useful to use diagrams as mneumonics for the integrals, and objects appearing in the integrals. The diagramatic expansion for the 
above solution to the integral propagator equation is

\ba
full\;propagator,\;G=
\parbox{20mm}{
\begin{fmffile}{feyn1}
\begin{fmfgraph}(40,25)
  \fmfleft{i}
  \fmfright{o}
  \fmf{plain}{i,o}
\end{fmfgraph}
\end{fmffile}}
+
\parbox{20mm}{
\begin{fmffile}{feyn2}
\begin{fmfgraph}(40,25)
  \fmfleft{i}
  \fmfright{o}
  \fmf{plain}{i,v}
  \fmf{plain}{v,o}
  \fmfdot{v}
\end{fmfgraph}
\end{fmffile}}
+
\parbox{20mm}{
\begin{fmffile}{feyn3}
\begin{fmfgraph}(40,25)
  \fmfleft{i}
  \fmfright{o}
  \fmf{plain}{i,v1}
  \fmf{plain,tension=0.5}{v1,v2}
  \fmf{plain}{v2,o}
  \fmfdot{v1}
  \fmfdot{v2}
\end{fmfgraph}
\end{fmffile}}+...
\ea
where the line represents the free propagator $G_0$ and the vertices (dots) represent the interaction term. We then think of a given diagram in physical terms as the particle moving for 
a bit, then interacting with the external potential, then moving a bit, and so on. Quantum mechanics is a single-particle theory, so the diagramatic technique is not really needed. But 
in the standard model we shall see that it allows for a much simpler visualization of what is going on.



%%%%%%%%%%%%%%%%%%%%%%%%%%%%%%%%%%%%%%%%%%%%%%%%%%%%%%%%%%%%%%%%%%%
%%%%%%%%%%%%%%%%%%%%%%%%%%%%%%%%%%%%%%%%%%%%%%%%%%%%%%%%%%%%%%%%%%%
%%%%%%%%%%%%%%%%%%%%%%%%%%%%%%%%%%%%%%%%%%%%%%%%%%%%%%%%%%%%%%%%%%%
%%%%%%%%%%%%%%%%%%%%%%%%%%%%
