\section{What holds a crystal together?}
\label{sec:bonding}

\footnotetext[1]{Useful texts for Part~\ref{sec:bonding} include
Chapter 1~\cite{Hook_and_hall,HartDavis,Blakemore}, Chapter
3~\cite{Kittel}, Chapter 14~\cite{Atkins} and the introductory
chapter of pretty much any solid state textbook (QC176)}

Solid state physics is entirely concerned with explaining the
properties of solid materials made up of closely interacting
atomic nuclei and electrons.  Our aim is to understand atoms arranged in well-defined crystalline structures, but before
we do, we should consider the forces that
attract the atoms to one another in the first place.  These
forces are responsible for the physical properties of everything
around us. As pointed out by
~\cite{Hook_and_hall} \emph{`All bonding is a consequence of the
electrostatic interaction between nuclei and electrons obeying
Schr\"{o}dinger's equation`}. However, by approximating bonding into categories where one physical phenomenon dominates, we can build accurate models of almost all materials and predict 
their physical and electronic properties. The types of interatomic bonding that are important to us in this module are: van der Waals, ionic, covalent and metallic.
